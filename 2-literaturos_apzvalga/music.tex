% !TEX root = ../gnss_interference_resistant_thesis.tex
\documentclass[main.tex]{subfiles}

\begin{document}

\subsection{Spindulio krypties nustatymas}\label{sec:music}

Apdorojant signalus iš antenų masyvų, galima ne tik atlikti spindulio
formavimą, tačiau ir aptikti siųstuvų padėtis antenų masyvo atžvilgiu.
Vienas iš signalų apdorojimo metodų yra MUSIC (angl. Multiple Signal Classification)
algoritmas kuris gali nustatyti kelių signalų kryptis.
Šis algoritmas buvo išrastas Schmidt \cite{1143830} ir nepriklausomai
Bienvenu ir Kopp \cite{1171029}, \cite{1164185}.

Signalo modelis aprašo imtuvo gautus duomenis per siųstuvo siunčiamą signalą. Tarkime, kad
turime $D$ nekoreliuotų ar dalinai koreliuotų signalų šaltinių $s_d(t)$. Imtuvo
duomenys $x_m(t)$ yra sudaryti iš signalų, priimtų antenų masyve, kartu su triukšmu
$n_m(t)$.

\begin{equation}
    x(t) = As(t) + n(t),
\end{equation}
\begin{equation}
    s(t) = [s_1(t), s_2(t),...,s_D(t)],
\end{equation}
\begin{equation}
    A = [a(\theta_1) | a(\theta_2) | ... | a(\theta_D)],
\end{equation}

\noindent čia:

\begin{itemize}
    \item $x(t)$ yra vektorius priimtų signalų iš antenų masyvų. Vektoriaus dydis yra $M$;
    \item $A$ - $M\times D$ matrica kurioje yra signalo priėmimo krypčių vektoriai. Vektorius
    yra sudarytas iš reliatyvių fazės skirtumų tarp antenų iš vieno signalo šaltinio. Kiekvienas
    $A$ stulpelis parodo atskirą signalo šaltinį su savo priėmimo kryptimi;
    \item $s(t)$ yra $D\times 1$ vektorius priimtų signalų antenų masyve;
    \item $n(t)$ imtuvo triukšmų vertės.
\end{itemize}

Svarbus kiekvieno poerdvinio metodo dydis yra sensorių kovariantiškumo matrica $R_x$.
Kai signalų triukšmai nėra koreliuoti, imtuvų kovariantiškumo matricos turi du komponentus,
signalų kovariantiškumo matricą ir triukšmų kovariantiškumo matricą:

\begin{equation}
    R_x = E{xx^H} = AR_sA^H + \sigma^2_n I,
\end{equation}

\noindent čia $R_s$ signalo kovariantiškumo matrica:

\begin{equation}
    R_s=E{ss^H}.
\end{equation}

\noindent Nekoreliuotiems ar dalinai koreliuotiems šaltiniams, $R_s$ yra Ermito matrica
ir turi pilną rangą $D$.

Taikant MUSIC algoritmą, darome prielaidą, kad triukšmų galia visuose imtuvuose yra vienoda
ir jie yra nekoreliuoti. Padarius tokią prielaidą, triukšmų kovariantiškumo matrica patampa
su lygiomis vertėmis įstrižainėje. Kadangi tikroji sensorių kovariantiškumo matrica
yra nežinoma, MUSIC algoritmas įvertina sensorių kovariantiškumo matricą $R_x$
iš priimto signalo kovariantiškumo matricos. Priimto signalo kovariantiškumo matrica
yra vidurkis kelių atskirų matavimų:

\begin{equation}
    R_x = \frac{1}{T} \sum^T_{k=1} {x(t)x(t)^H}.
\end{equation}

Kadangi matrica $AR_sA^H$ turi rangą $D$, ji turi $D$ teigiamų tikrinių verčių ir
$M-D$ nulinių tikrinių verčių.
Tikriniai vektoriai turintys teigiamas vertes padengia signalų poerdvį $U_s=[v_1,v_2,...,v_D]$.
Tikriniai vektoriai turintys nulines vertes yra statmeni signalo poerdviui ir apimą
nulinį poerdvį $U_n=[u_{D+1},...,u_N]$. Nuliniai poerdvio vektoriai turi tenkinti
šią sąlyga:

\begin{equation}
    AR_sA^Hu_i=0 \Rightarrow u^HAR_sA^Hu_i = 0 \Rightarrow (A^Hu_i)^HR_s(A^Hu_i)=0 \Rightarrow A^Hu_i=0.
\end{equation}

Pridėjus triukšmus prie tikrinių signalo vektorių sensorių kovariantiškumo matricos, ji
išlieka nepakitusi. Prie tikrinės vertės prisideda triukšmų galia. Tarkime
$v_i$ vienas iš betriukšmių signalo poerdvio tikrinių vektorių:

\begin{equation}
    R_xv_i=AR_sA^Hv_i+\sigma^2_0Iv_i = (\lambda_i + \sigma^2_0)v_i.
\end{equation}

\noindent Nulinio poerdvio tikrinis vektorius taip pat yra $R_x$ matricos tikrinis
vektorius. Tarkime $u_i$ yra vienas iš nulinių tikrinių vektorių:

\begin{equation}
    R_xu_i=AR_sA^Hu_i+\sigma^2_0Iu_i+\sigma^2_0Iu_i=\sigma^2_0u_i,
\end{equation}

\noindent tikrinėms vertėms esant $\sigma^2_0$ vietoje nulinės vertės, nulinis
poerdvis patampa triukšmų poerdviu.

MUSIC algoritmas veikia paieškos principu, ieškant visų priėmimo vektorių kurie
yra statmeni triukšmų poerdviui. Tam kad atlikti paiešką algoritmas sukuria
nuo priėmimo kampo priklausantį galios išraišką, vadinamą MUSIC pseudospektru:

\begin{equation}
    P_{MUSIC}(\overrightarrow{\phi}) = \frac{1}{a^H(\overrightarrow{\phi})U_nU_n^Ha(\overrightarrow{\phi})}.
\end{equation}

\noindent Kai signalo vektorius yra statmenas triukšmų poerdviui, pseudospektro maksimumai
yra begaliniai. Taikant algoritmą praktiškai, vektoriai niekada nėra visiškai statmeni,
todėl sklidimo kampai yra nustatomi iš $P_{MUSIC}$ maksimumų.

\end{document}