% !TEX root = ../gnss_interference_resistant_thesis.tex
\documentclass[main.tex]{subfiles}

\begin{document}

\subsection{Programinis radijas}

Programinis radijas (SDR, angl. Software-defined radio) yra komunikacijos sistema,
kurios dauguma komponentų yra įgyvendinta programiškai. Analoginė radijo dalis
atlieka tik IQ demoduliaciją, demoduliuoto signalo skaitmenizavimą.
Visi kiti radijo komponentai (filtrai, demoduliatoriai, detektoriai) yra įgyvendinti
pasitelkiant programinę įrangą.

Pagrindiniai programinio radijo privalumai \cite{Sadiku-2004}:
\begin{itemize}
    \item lankstumas;
    \item lengvas pritaikymas;
\end{itemize}

\subsubsection{IQ moduliacija}

IQ moduliatorius yra centrinė SDR dalis. Norint apdoroti primatą RF signalą, reikia
jį perkelti į žemesnių dažnių juostą. Tai pasiekiama sudauginant gaunamą signalą su dviem
sinusoidėmis, kurių fazė skiriasi $\pi/2$, taip gaunamas $x_I(t)$ ir $x_Q(t)$ signalai,
kaip pavaizduota \ref{fig:iq_modulator}~pav.

\begin{figure}[h]
    \begin{centering}
    \includegraphics[scale=1.0]{drawings/iq_modulator}
    \par\end{centering}
    \protect\caption{\label{fig:iq_modulator}IQ demoduliatoriaus schema}
\end{figure}

\subsubsection{SDR blokinė diagrama}

Pagrindinė SDR radijo užduotis, yra demoduliuoti signalą, jį suskaitmenizuoti, ir perduoti
IQ signalą programinei įrangai. SDR siųstuvas susideda iš kelių pagrindinių dalių:

\begin{enumerate}
    \item RF signalų paruošimo grandinė (stiprintuvas, filtrai);
    \item IQ demoduliatorius;
    \item ADC (angl. Analog-Digital converter);
    \item Mikroprocesorius;
    \item Osciliatorius;
\end{enumerate}

\begin{figure}[h]
    \begin{centering}
    \includegraphics[scale=0.85]{drawings/sdr_blockdiagram}
    \par\end{centering}
    \protect\caption{\label{fig:sdr_blockdiagram}SDR radijo blokinė diagrama}
\end{figure}

RF signalų paruošimo grandinė (1) yra atsakinga už signalo sutvarkymą, kad jis
būtų tinkamas demoduliacijai. Dažniausiai atliekamas signalo stiprinimas,
filtravimas, taip pat būna integruotas maitinimas aktyviai antenai. ADC (3) atlieka IQ
signalo skaitmenizavimą, suskaitmenizuotas signalas yra perduodamas mikroprocesoriui (4),
kuris arba atlieką skaitmeninį signalų apdorojimą, arba juos perduoda kitam įrenginiui.
Taip pat mikroprocesorius yra atsakingas už visų grandinių darbo valdymą: parenka stiprinimo
koeficientus, valdo osciliatorių, parenka RF kelią pagal nustatytus sistemos reikalavimus.
Osciliatorius (5) atsakingas už įvairų taktinių dažnių generavimą.

\end{document}




