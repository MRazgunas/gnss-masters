% !TEX root = ../gnss_interference_resistant_thesis.tex
\documentclass[../gnss_interference_resistant_thesis.tex]{subfiles}

\begin{document}

\section*{Įvadas}\addcontentsline{toc}{section}{Įvadas}

Bepiločių orlaiviai, kaip pagrindinį navigacijos instrumentą naudoja
GNSS sistemą. Esant geroms sąlygoms, GNSS sistema veikia pakankamai
patikimai, kad būtų užtikrinta tiksli pozicija,
tačiau mėginant naudoti orlaivius, aplinkose kur yra daug
pašalinių objektų, dėl atsiradusių GNSS paklaidų, pasikliauti gaunamais
duomenimis nebegalima.

Didžiausią problemą GNSS tikslumui, miesto teritorijose, kelia
atspindžiai nuo pastatų. Dėl šių atspindžių, prastėja
koreliacijos rezultatas, dėl to gaunama mažiau patikima
pozicija \cite{Vagle2016}. Vienas iš galimų metodų,
pašalinti atspindžių įtaką yra atsispindėjusio signalo
silpninimas, pasinaudojus antenų masyvu (angl. phased antenna array).
Šiame darbe bus tyrinėjama antenų masyvo konstrukcija,
valdymas, pasinaudojant nebrangiais SDR imtuvais.

Naudojant antenų masyvą, galima ne tik formuoti spindulį,
tačiau ir išgauti informaciją apie signalo priėmimo kryptį (DOA, angl. direction of arrival).
Ši informacija leistų aptikti natūralius ir dirbtinius trikdžius.
Žinant signalų kryptį ir pasinaudojus spindulio formavimo algoritmu
galima nuslopinti trikdį suformuojant nulį trikdžio kryptimi
arba suformuojant maksimumą tikrojo signalo kryptimi.
Signalo krypties informacija taip pat leidžia autentifikuoti palydovo signalą,
kadangi orbitos yra tiksliai žinomos. Gaunant signalą kita kryptimi
negu tikimasi, galima jį nuslopinti ir naudoti tikrąjį palydovo signalą.

Šio darbo tikslas - sukurti trikdžiams atsparią GNSS sistemą,
kurią būtų galima pritaikyti bepiločių orlaivių navigacijoje miesto
teritorijose.

Darbas atliktas vykdant projektą
"Bepiločio orlaivio, skirto aptikti ir nukenksminti bepiločius orlaivius kūrimas (SMART)"
Nr. 01.2.2-LMT-K-718-01-0029.

\end{document}
