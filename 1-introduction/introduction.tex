% !TEX root = ../gnss_interference_resistant_thesis.tex
\documentclass[../gnss_interference_resistant_thesis.tex]{subfiles}

\begin{document}

\section*{Įvadas}\addcontentsline{toc}{section}{Įvadas}

Bepiločių orlaiviai, kaip pagrindinį navigacijos instrumentą naudoja
GNSS sistemą. Esant geroms sąlygoms, GNSS sistema veikia pakankamai
patikimai, kad būtų užtikrinta tiksli pozicija,
tačiau mėginant naudoti orlaivius, aplinkose kur yra daug
pašalinių objektų, dėl atsiradusių GNSS paklaidų, pasikliauti gaunamais
duomenimis nebegalima.

Didžiausią problemą GNSS tikslumui, miesto teritorijose, kelia
atspindžiai nuo pastatų. Dėl šių atspindžių, prastėja
koreliacijos rezultatas, ko pasėkoje gaunama mažiau patikima
pozicija \cite{Vagle2016}. Vienas iš galimų metodų,
pašalinti atspindžių įtaką yra, atsispindėjusio signalo
silpninimas, pasinaudojus antenų masyvu (angl. phased antenna array).
Šiame darbe bus tyrinėjama antenų masyvo konstrukcija,
valdymas, pasinaudojat nebrangiais SDR imtuvais.

Šio darbo tikslas - sukurti trikdžiams atsparią GNSS sistemą,
kurią būtų galima pritaikyti bepiločių orlaivių navigacijoje, miesto
teritorijose.

\end{document}
