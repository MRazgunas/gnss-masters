% !TEX root = ../gnss_interference_resistant_thesis.tex
\documentclass[../gnss_interference_resistant_thesis.tex]{subfiles}

\begin{document}

\section*{Išvados ir rezultatai}\addcontentsline{toc}{section}{Išvados ir rezultatai}

\begin{enumerate}
    \item Tiriant HackRF laiko sinchronizaciją tarp kelių imtuvų, nustatyta, kad gauta sinchronizacija
          geresnė negu $\pm 1$ nuskaitymo taškas. Atliekant tuos pačius matavimus su KerberosSDR imtuvu,
          gauta, kad vėlinimas tarp imtuvų yra kintamas, todėl šiam imtuvui reikalingas vėlinimo kalibravimas.
    \item Tiriant HackRF ir KerberosSDR fazinę sinchronizaciją tarp kelių imtuvų, nustatyta, kad fazės skirtumas
          tarp imtuvų nusistovi greičiau negu per $5\ \mathrm{s}$, HackRF imtuvo fazė kinta bėgant laikui, per
          20 minučių pakinta apie $0,35\ \mathrm{rad}$, o KerberosSDR imtuvo fazė yra stabili ir nekinta.
          Norint panaudoti HackRF imtuvus, reikalingas periodiškas fazės kalibravimas.
    \item Išmatavus spinduolių sistemos kryptingumo diagramą, nustatyta, kad gaunamas $3\ \mathrm{dB}$
          signalo padidėjimas, naudojant spindulio formavimą ir pademonstruotas spindulio krypties keitimas
          pasinaudojant sinchronizuotus HackRF imtuvus.
\end{enumerate}

\end{document}
