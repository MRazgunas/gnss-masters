% !TEX root = ./gnss_interference_resistant_thesis.tex
\documentclass[./gnss_interference_resistant_thesis.tex]{subfiles}

\begin{document}

\santrauka{R\&D of Robust to Interferences GNSS Navigation Subsystem for UAVs}{

\iffalse
Terahertzų (100GHz - 10THz) sritis šiuo metu yra sparčiai vystoma ir tyrinėjama, dėl plačių jos taikymų.
Parodžius,
kad THz bangas gali detektuoti lauko tranzistoriai, atsirado galimybės gaminti detektorius ir jų matricas
pasinaudojus masinės gamybos technologijomis. Perkonfigūruojamos jutiklių matricos leidžia nuskaityti spindulio
profilius, vaizdus. Sujungiant kelis jutiklius į vieną galima sukurti jutiklį pritaikytą norimam spindulio
profiliui.

Šiame darbe buvo pademonstruotas THz jutiklių matricos valdymo ir nuskaitymo metodas
leidžiantis nesunkiai didinti jutiklių matricas. Darbe apibūdinta valdymo elektronikos veikimo principai,
skaitmeninių potenciometrų panaudojimas užtūros įtampos valdymui, mažatriukšmių stiprintuvų dizainas. Parodyta,
kad valdymo elektronika neįneša reikšmingų papildomų triukšmų į matuojamą signalą.

Pasinaudojus sukurta valdymo elektronika buvo ištyrinėta AlGaN/GaN 7x6 jutiklių matrica,
buvo nustatyta, kad perkonfigūruojant jutiklių matrica, buvo pasiektas $3,3\mathrm{dB}$ SNR padidėjimas.
Išmatuotos detektoriaus charakteristikos: varžos, NEP, jautrio priklausomybės nuo užtūros įtampos, dažninės
jautrio ir NEP priklausomybes intervale nuo $50\mathrm{GHz}$ iki $1200\mathrm{GHz}$. Nustatytas naudojamų detektorių geriausias jautris $115,2\mathrm{V/W}$ ir NEP $93\mathrm{pW/\sqrt{Hz}}$ prie $278\mathrm{GHz}$.
\fi

\foreignlanguage{english}{
Terahertz (100GHz - 10THz) technologies is undergoing rapid research and development due to their potential to wide range of
applications. It was recently shown that THz radiation can be efficiently detected by field effect transistors. This enables large scale production
of detectors by employing well developed semiconductor fabrication technologies. 
One of new opportunities for applications brought by these technologies is a possibility to implement
detector with configurable effective cross section.
Such detector could allow to scan beam profile, one can configure a detector adapted to the desired beam
cross section.
}

\foreignlanguage{english}{
This work demonstrates THz detector matrix control and scanning method, allowing easy expansion to bigger
matrices. This work describes principles of operation of control electronics, use of digital potentiometers
for gate voltage control, low noise amplifier design. Designed control circuitry does not introduce a significant amount of noise to the measured signal.
}

\foreignlanguage{english}{
Using developed control electronics , the matrix of AlGaN/GaN 7x6 sensor array was studied. It was found
that an increase of SNR by $3,3\ \mathrm{dB}$ was achieved by reconfiguring detector matrix. The most important 
characteristics of the detector were studied: resistance, sensitivity and NEP dependence to gate voltage.
Sensitivity and NEP frequency characteristics was measured in range from $50\ \mathrm{GHz}$ to $1200\ \mathrm{GHz}$.
Best sensitivity achieved was $115,2\ \mathrm{V/W}$ and NEP $93\ \mathrm{pW/\sqrt{Hz}}$ at $278\ \mathrm{GHz}$.
}
}

\end{document}