% !TEX root = ../gnss_interference_resistant_thesis.tex
\documentclass[../gnss_interference_resistant_thesis.tex]{subfiles}

\begin{document}

\section{Trikdžiams atsparios palydovinės navigacijos sistemos tyrimas}

Visų matavimu metu naudojamas nešlio dažnis
$f_D = 1575,42\ \mathrm{MHz}$, nuskaitymo dažnis $f = 2\ \mathrm{MHz}$.
Kaip $10\ \mathrm{MHz}$ šaltinis
naudojamas OLP01-12s-MF5.8-36A, temperatūriškai stabilizuotas osciliatorius,
kurio specifikacijos pateiktos \ref{tab:clock_source_spec} lentelėje.


\begin{table}[h]
    \protect\caption{\label{tab:clock_source_spec}OLP01-12s-MF5.8-36A specifikacijos \cite{macrobizes_olp}.}
    \centering{}%
    \begin{tabular}{| c | c |}
    \hline
    Parametro pavadinimas & Vertė \\
    \hline
    Dažnis                         & $10\ \mathrm{MHz}$ \\
    Temperatūrinis stabilumas      & $\pm \num{5e-8}\ \mathrm{Hz/\degree C}$ \\
    Maitinimo įtampos stabilumas   & $\pm \num{2e-9}\ \mathrm{Hz/V}$ \\
    Laikinis stabilumas             & $\pm \num{5e-8}\ \mathrm{Hz/Metus}$ \\
    \hline
    \end{tabular}
\end{table}

\subfile{time_sync.tex}
\subfile{phase.tex}
\subfile{time_sync_kerberos.tex}
\subfile{phase_kerberos.tex}
\subfile{beam_form_meas.tex}
\subfile{gnss_measurement.tex}

\end{document}
