% !TEX root = ../gnss_interference_resistant_thesis.tex
\documentclass[../gnss_interference_resistant_thesis.tex]{subfiles}

\begin{document}

\santrauka{R\&D of Robust to Interferences GNSS Navigation Subsystem for UAVs}{

\iffalse
Bepiločių orlaiviai, kaip pagrindinį navigacijos instrumentą naudoja GNSS sistemą. Esant
geroms sąlygoms imtuvas veikia pakankamai patikimai, kad būtų užtrikrintas saugus bepiločio
orlaivio veikimas. Tačiau didėjant bebiločių orlaivių taikymo sritims, vis dažniau jie yra
naudojami aplinkose su daugybe įvairių trikdžių.
Vienas dažniausiais pasitaikančių trikdžių miesto teritojiose yra atspindžiai nuo pastatų.
Dėl šių atspindžių prastėja GNSS signalo koreliacijos rezultatas, dėl ko gaunama pozicija su
mažesniu tikslumu.

Pastaraisiais dešimtmečiais plačiai pradėti taikyti antenų masyvai,
leidžia nuslopinti trigdžius priimama signalui, todėl šiame darbe buvo sukurtas
GNSS imtuvas kuris pasinaudodamas spindulio formavimo metodais, slopina atspindžių
trukdžius. Sukonstruotas imtuvas geba aptikti GNSS signalo sklidimo kryptis pasinaudodamas
MUSIC algoritmu, bei suformavus spindulį priimti palydovų signalus be atspindžių.
Imtuvo konstrukcijai buvo panaudoti HackRF imtuvai sinchronizuoti laike bei dažnyje.

Šiame darbe buvo tiriamas sukonstruoto imtuvo laikinės ir dažninės sinchronizacijos, pademonstruota,
kad HackRF imtuvai gali būti pritaikomi spindulio formavimui.
Ištyrus sukurtą GNSS imtuvą, buvo pademnstuotas signalo ir atspindžių aptikimas skirtingose matavimų vietose:
vietovėje be atspindžių, vietovėje su nedideliais atspindžių šaltiniais, bei urbanistinėje miesto
vietovėje. Taikant spindulio formavimą buvo pademonstuotas signalo triukšmo santyko padidėjimas iki
$8 \mathrm{dB}$, atspindžių slopinimas, bei imtuvo koordinatės pasiskirstymo sumažėjimas.
\fi

Unmanned Arial Vehicles (UAV) utilize GNSS receivers as the main navigation instrument, in
ideal conditions receiver works sufficiently well to ensure safe operation of UAV.
In recent years application fields of UAVs are expanding, and more often, they are used in environments
with various interference to GNSS signals. One of the most common interference sources
is a reflection from buildings in urbanized territories, which decreases performance often
GNSS receiver correlator, which degrades the performance of receiver.


In recent decades array processing has been adopted to suppress interference.
This work created a GNSS receiver, which can attenuate reflection interference
by utilizing antenna array processing. 
The receiver can distinguish the direction of signal arrival using the MUSIC algorithm,
and applying beamforming techniques can improve navigation precision.
The receiver is constructed with HackRF SDRs, synchronized in time and frequency.

In this work constructed GNSS receiver was studied. It was shown that it is possible
to synchronize HackRF SDRs, which allows applying MUSIC and beamforming algorithms.
The receiver was studied in different environments: open field, low reflection, and urbanistic with tall buildings.
It was shown that signal to noise ratio
is improved up to $8\ \mathrm{dB}$ by applying beamforming. Also, it was revealed that
by using MUSIC, it is possible to detect the angle of arrival of GNSS signals and also,
it is possible to see reflected signals; when array processing was applied,
deviation of receiver coordinates was improved.

}

\end{document}